\section{Design}

Make it possible for the everyday user to install their application.

\subsection{Components}

Monitor directories with SDB.

Principles of a secure system: Authenticity, Integrity,
Confidentiality.

\subsection{Renaming Files}

Identify files based on inode numbers, md5, and mtime.

\subsubsection*{Hybrid cryptosystem}
Encrypt files with block cipher (AES-256). Encrypt the password as a
file with public keys (RSA).  Upload the AES-encrypted and the public
key-encrypted password (one public-key encrypted password per user, of
course).

SDB provides a high degree of privacy, understood here as freedom from
external inteference. In existing systems, one of these most common
and practical methods for securely storing data is using a symmetric
key encryption method called AES-256. Symmetric key cryptography is
very similar to one's everyday use of passwords to log into computers,
email services, etc.; to access a password-encrypted document, one
just needs to know the correct password. The important point is that
the ``key'' must be shared between all parties who will want to have
access to the data. Secret sharing techniques are applicable but we
think there might be a more efficient approach leveraging the
cloud. However, services like Dropbox raise a concern over middleman
companies that use symmetric key encryption since the middleman stores
the password that is used to encrypt the stored data. The middleman
can see the data and can release the data when needed or desired.

Public key cryptographic techniques in practice are used to sign and
encrypt data for two purposes: 
\begin{enumerate}
\item Signing data (with the private key) guarantees that the
  recipient knows that sender of the data is who the sender claims to
  be.
\item Encrypting data (with the public key) guarantees that only the
  corresponding private key owner can decrypt the data. Notably, the
  public key cannot be used to decrypt the public key encrypted data.
\end{enumerate}

SDB works using a well-known combination of symmetric key and private
key cryptography to ensure true privacy and security for user’s
files. A randomly generated 32 byte password is generated per
file. This password is used to AES-256 encrypt a file. The password is
then encrypted with the public key of whomever the file owner wants to
share the file with. The public key encrypted password files (plural
since we have multiple users we want to share the file with) and the
symmetric key encrypted file are uploaded to the cloud service. The
symmetric key encrypted file and the corresponding public key
encrypted password file are made accessible to the user the file owner
wants to share the original data with.

\subsection{Address Book}

Users maintain their contacts and locations in an \name Address
Book. The address book is for \name users and maps an email address to
(location, public key) pairs.

\subsection{Definitions}

\begin{itemize}
\item deltas
\item hashes
\item checkpoint files
\item SimpleDB
\item SQLite
\item S3
\end{itemize}

\subsection{Order of Operations}

Mixing our hybrid cryptosystem with delta encoding requires careful
ordering of algorithms to ensure sufficient entropy in our system.

We always perform delta encoding operations on cleartext, encrypting
hashes, deltas, and checkpoint files after they contain the proper
information.



