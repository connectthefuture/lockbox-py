\begin{abstract}

  To scale to millions of page requests per day, many popular web services such
  as Facebook, eBay, and Amazon partition application data, storing each
  partition on a separate database instance running on its own machine.  Since
  these user-facing web services have strict latency and availability
  requirements, it costs too much to run distributed commit protocols to support
  ACID transactions across data partitions stored on different machines.
  Programmers can decompose an application operation into several smaller
  transactions each modifying a single data partition; however, programmers lack
  tools to help automatically cope with failures that occur between multiple
  transactions.  This paper introduces the abstraction of an Atomic Transaction
  Chain (\atc), which allows programmers to group a series of related
  transactions that modify different data partitions under all-or-nothing atomic
  execution.  We also present the design and implementation of \name, which provides a SQL
  extension to simplify operations on a partitioned database. In particular,
  \name allows programmers to perform primary key-based lookups/updates across
  partitioned tables, specify custom {\atc}s, and create
  secondary indices that are automatically maintained by the underlying storage
  system through its use of \atc.

	We have ported a third-party application RuBIS~\cite{rubis} (an online auction website) to
	run on top of \name. Evaluations on a local cluster testbed show that \name
	achieves good scaling performance across many machines.

\end{abstract} 
