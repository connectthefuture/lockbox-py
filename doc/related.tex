\section{Related Work}

\subsection{W5 Realized?}
W5~\cite{w5} summarized the idea of enabling users to store their data
on the cloud without having to reveal information about themselves to
the cloud itself.

Ellis, et al.~\cite{groupware} introduced the notion of collaborative
groupware and the challenges groupware poses to concurrency.

\subsection{Delta Encoding}
The idea of applying rolling checksums for the purposes of delta
encoding has been widely deployed in the utility {\tt
  rsync}~\cite{rsync}. Algorithmically, these ideas relate to Rabin
fingerprinting~\cite{rabinfp}.

The Low-Bandwidth File System~\cite{lbfs} has demonstrated the utility
of rolling checksums and caching on a file system level. This
complicates the matter of making it easy for users to adopt our
utility. While algorithmically sufficient, users finding adoption
(i.e., configuring a new file system) too difficult.

\subsection{Versioning Provenance}

Braun, et al.~\cite{securingprovenance} acknowledge the difference in
access control management strategies for provenace being the
relationships between data: How do you secure provenance correctly and
coherently for the end-user?

\subsection{Sharing and Key Management}
Shamir explained how to share a secret~\cite{shamir}.\note{beyond the
  gratuitous ack...}

Attribute-based encryption shows how we can use identity-based
encryption techniques to provide access control over encrypted
data~\cite{abe}. \note{How in particular?}

\subsection{(Dis)Trusting the Cloud}
Oceanstore~\cite{oceanstore} showed the design for a secure,
distributed file system. Pond~\cite{pond} demonstrated an early
prototype of the system. Use of the cloud simplifies many of our
design considerations. We believe that these techniques are still
valid and perhaps influenced the internal design of cloud service
providers on which we tested our system.

Bayou~\cite{bayou} presents an evaluation of conflict resolution that
occurs at the mobile device user level. In our system, we actually
leverage the consistency models of existing web service providers (S3
has atomic writes but eventually consistent reads of raw data while
SimpleDB has the option to enable consistent reads of metadata.)

Distrusting the cloud has been a common thread of research for several
years. Cachin, et al.~\cite{trusting} review the methods by which one
could come to trust the cloud. Mazieres, et al.~\cite{sfsbyzantine,
  donttrust} discuss the design considerations behind the realization
of a file system that does not have to trust the storage
service. SiRiUS~\cite{sirius} also wants to distrust the cloud
greatly, especially teh networking. Depot~\cite{depot} summarizes and
introduces the various consistency models. SPORC~\cite{sporc}
introduced the question: What if we don't trust the cloud to order our
messages correctly? Starting with this question, overhead enters the
picture. A point of comparison in our system is whether such a design
consideration presents an unnecesary overhead in designing a
cloud-based user application. SUNDR~\cite{sundr}.

When considering the trust worthiness of our history
mechanisms~\cite{fakepicasso}.

\note{gratuitous additional thoughts:} Hippocratic
Databases~\cite{hippocratic} and searchable encrypted databases that
reveal only answers to queries~\cite{dawn}.

Arguably, the trustworthiness of the cloud is not as bad as these
papers make the situation out to be.

